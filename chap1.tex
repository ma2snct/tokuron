\subsection{国内の医療情報共有の現状}
 医療の連携はうまくいっていない.ICTでいいかんじにやろうと国主体でやってるが,いまいち.あじさいネットは成功例.でも全国に普及してるわけではない.\cite{bibi3}

\subsection{企業発信の類似製品}
 共通の規格が活用されていない現実があるので,いろんな規格の差を吸収できるようなアプリは需要があるんじゃないかな.とりあえずシェアが大きそうなss-mixを中心に
 既存アプリid-linなどは患者idをリンクしているだけで情報を一元的に集約はしていない.

\subsection{地方のかかりつけ医のニーズ}
 ,将来の医療情報共有のコンセプトを提案する .様々なフォーマットを医療関係者,患者の二者からの入力を受け付ける(患者からしか入力できないどこでもmy病院との差別化).

\subsection{SQL版について}
 医療大から要望があったエクセル形式のデータについてのアプリはDjangoで開発を終えた.今後需要があるであろうバイタルデータの活用に向けて,NoSQLを用いたアプリ開発を行う.

