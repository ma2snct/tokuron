\subsection{背景}
  現在の日本の医療システムにおいて,手術を必要とする病気にかかった場合,
  患者は手術のため大病院と,経過観察のためかかりつけ医の間を
  何度か移動することがある.
  このとき,これらの病院で重複する検査や診断を受けることがある.
  近年の電子カルテの普及により,医療情報の電子化は進んでいるが,
  それは病院ごとに個々に管理されている.
  診断時の患者の状態を把握する必要がある場合,
  双方の病院において検査などを行う必要があるが,
  単に情報が得られればよい場合,
  一方の医療機関に存在する情報を別の医療機関で
  改めて作ることは医療コストの無駄である.
  必要な情報が共有されることにより
  患者や医療関係者の負担が減ることが予想される.
  しかし,現在,国内には患者の医療情報を統一して共有する
  システムがないため,
  医療情報は各病院で電子カルテにより電子化されていたとしても,
  情報共有は口伝えや紹介状に留まる.
  国内で利用されている電子カルテは標準規格がないまま各企業において開発されたため,
  規格にばらつきがあり病院間で共有することは難しい.\cite{bibi2}

  また,最近はスマートフォンのヘルスアプリや家庭用血圧計などから
  個人が自身の医療情報としてバイタルを生成することができる.
  スマートフォンを使うことにより患者が意識することなく,
  バイタルのログを記録することができ,
  さらにその記録周期を短くすることにより,さまざまな情報を得ることができることが考えられる.
  これらの情報を患者自身の定常時のバイタルとして持つことで,通院,
  入院時の状態と比較したり,
  異常の早期発見につながったりすることが考えられる.

  そこで本研究では,限定された地域内の患者,複数の医療機関の医者
  で医療情報を共有するための環境構築を目指し,
  システムのプロトタイプを開発した.

\subsection{本論文の構成}
  2章では日本の医療情報の共有に関する情報を記載している.
  3章では開発するシステムの目標とそれを実現するための設計について説明する.
  4章5章では異なる機能に着目したシステムの開発について述べる.
  6章では開発したシステムの考察を述べる.
