\subsection{医療情報の形式}
  2015年現在,日本国内には電子カルテの出力ファイルや,
  医療関係者が独自に生成したエクセルファイルまで
  様々な形式で医療情報が電子化されている.
  主な例を以下に示す.

  \subsubsection{HL7}
  HL7とはHealth Level Sevenの略称である.
  医療情報システム間のISO-OSI第7層アプリケーション層に由来している.
  2015年11月現在,国内で約20の企業が会員となっている.
  特定の部門やシステムに特化したものでなく,施設間・システム間での
  臨床情報や管理情報を扱い,相互運用性を高めるための
  ヘルスケア領域でのデータ交換標準である.\cite{bibi5} \cite{bibi6}

  データ定義は図\ref{ss-mix_sample}のようになっている.
  また,図\ref{ss-mix_sampledata}が
  その出力データのサンプルである.

	\begin{figure}[htbp]
    \begin{center}
			\includegraphics[width=18cm, bb=0 0 792 630]{./gazou/ss-mix_sample.png} %よこたて
    \end{center}
    \caption{HL7のデータ定義}
		\label{ss-mix_sample}
	\end{figure}

	\begin{figure}[htbp]
    \begin{center}
			\includegraphics[width=12cm, bb=0 0 688 437, clip]{./gazou/ss-mix_sampledata.png}
    \end{center}
    \caption{HL7の出力データの例}
		\label{ss-mix_sampledata}
	\end{figure}



  \subsubsection{かかりつけ医が生成するエクセルファイル}
  本研究のために鈴鹿医療科学大学から三重県内の
  ある医療機関の診療情報がエクセルのファイルで提供された.
  その検査値のエクセルファイルの例が図\ref{excel-data}である.
  また投薬のエクセルファイルの例が図\ref{excel-data-medicine}である.


  \begin{figure}[htbp]
    \begin{center}
      \includegraphics[width=13cm, bb=0 0 1049 733]{./gazou/excel-data.png}
    \end{center}
    \caption{検査値のエクセルファイルの例}
    \label{excel-data}
  \end{figure}

  \begin{figure}[htbp]
    \begin{center}
      \includegraphics[width=13cm, bb=0 0 850 411]{./gazou/excel-data-medicine.png}
    \end{center}
    \caption{投薬のエクセルファイルの例}
    \label{excel-data-medicine}
  \end{figure}



\subsection{本研究の類似製品,活動}
  国内の主な医療情報の共有に関する製品と活動を以下に示す.

  \subsubsection{ID-Link}
    ID-Linkは地域内の病院の患者IDを一元管理することで,
    地域内の病院で作られた電子カルテを参照することができるシステムである.
    医療情報そのものは収集していない.データセンターには患者IDのリンクが
    あるだけで,病院間を安全な通信技術で結び,相互参照させている.
    患者には事前に情報共有に関する許可をもらうことが通例になっている.
    シェアは2015年2月末に全国で4300の機関である.
    \cite{bibi12}

  \subsubsection{SS-MIX}
    SS-MIXは医療情報を収集するために,平成18年から動き出した
    厚生労働省を中心としたプロジェクトである.
    これは標準規格がないまま立ち上がった電子カルテの医療情報の
    電子化についての標準規格である.SS-MIXで規格化された
    基本情報,処方歴,検査結果を各機関のストレージに収集する.
    診断時に医師用端末から参照することや,紹介情報を作るときにも
    情報を引き出すことができる.
    これにはHL7が採用されている.
    \cite{bibi1}\cite{bibi7}

  \subsubsection{あじさいネット}
    2004年に長崎県大村市で始まり,2012年には,
    県域をカバーする地域医療連携ネットワークとして発展してきた.
    2013年4月現在において,電子カルテなどの患者情報の提供を
    行う地域の機関的病院は17病院,地域の診療所や調剤薬局などの
    情報閲覧施設は178施設,医療関係者の会員数は285名を数え,
    これまでに同意を得て登録された患者数は2万6千人を超えている.

    あじさいネットは10年にわたる活動の中でアンケートを繰り返し,
    会費だけで運用することができるシステムになっていった.
    \cite{bibi3}

\subsection{NoSQLデータベース}
NoSQLとはリレーショナルデータベースシステム(RDBMS)以外の
データベース管理システムのことである.
NoSQLの特徴として非構造データの取り扱いが得意であり,
水平分散が得意である.
RDBMSでは強い整合性を保つため水平分散が苦手である.

NoSQLには大きく3つの分類がある.
ひとつはKVSでキーバリューのデータ構造のものである.
次にグラフ型でノードと関連のそれぞれがキーバリューの
データ構造をもつものである.
最後にドキュメント型はデータ構造を階層的に管理できる.
ドキュメント型の製品にCouchDBがある.
CouchDBはデータベース内に複数のドキュメントを持つことで
データを管理している.
また,データをJSONで扱うのでデータ構造の柔軟性が高く,
事前にドキュメント構造を定義する必要がなく,
CouchDBを利用するシステム側の仕様変更などの都合に合わせて
データ構造を変更することができる.
\cite{bibi121}

これらの利点から本研究ではドキュメント型のNoSQLからCouchDBを用いた.
