\documentclass[12pt]{sotsuron}
\usepackage{graphicx}
\usepackage{wrapfig}

\title{患者が主体となった医療情報データベースシステムの開発}
\etitle{English Title}
\author{松岡 竜嗣}
\teacher{青山 俊弘 准教授}
\date{平成27年2月1日}
\affiliation{電子機械工学専攻}

\begin{document}
\maketitle
\begin{abstract}

BY EXCITE TRANSLATION
\end{abstract}

\pagenumbering{roman}
\tableofcontents
\clearpage

\pagenumbering{arabic}
\section{背景}
\subsection{背景}
  現在の日本の医療システムにおいて,手術を必要とする病気にかかった場合,
  患者は手術のため大病院と,経過観察のためかかりつけ医の間を
  何度か移動することがある.
  このとき,これらの病院で重複する検査や診断を受けることがある.
  近年の電子カルテの普及により,医療情報の電子化は進んでいるが,
  それは病院ごとに個々に管理されている.
  診断時の患者の状態を把握する必要がある場合,
  双方の病院において検査などを行う必要があるが,
  単に情報が得られればよい場合,
  一方の医療機関に存在する情報を別の医療機関で
  改めて作ることは医療コストの無駄である.
  必要な情報が共有されることにより
  患者や医療関係者の負担が減ることが予想される.
  しかし,現在,国内には患者の医療情報を統一して共有する
  システムがないため,
  医療情報は各病院で電子カルテにより電子化されていたとしても,
  情報共有は口伝えや紹介状に留まる.
  国内で利用されている電子カルテは標準規格がないまま各企業において開発されたため,
  規格にばらつきがあり病院間で共有することは難しい.\cite{bibi2}

  また,最近はスマートフォンのヘルスアプリや家庭用血圧計などから
  個人が自身の医療情報としてバイタルを生成することができる.
  スマートフォンを使うことにより患者が意識することなく,
  バイタルのログを記録することができ,
  さらにその記録周期を短くすることにより,さまざまな情報を得ることができることが考えられる.
  これらの情報を患者自身の定常時のバイタルとして持つことで,通院,
  入院時の状態と比較したり,
  異常の早期発見につながったりすることが考えられる.

  そこで本研究では,限定された地域内の患者,複数の医療機関の医者
  で医療情報を共有するための環境構築を目指し,
  システムのプロトタイプを開発した.

\subsection{本論文の構成}
  2章では日本の医療情報の共有に関する情報を記載している.
  3章では開発するシステムの目標とそれを実現するための設計について説明する.
  4章5章では異なる機能に着目したシステムの開発について述べる.
  6章では開発したシステムの考察を述べる.


\section{関連研究,従来DB}
\subsection{知識}
  2015年現在,日本国内には電子カルテの出力ファイルや,
  医療関係者が独自に生成したエクセルファイルまで
  様々な形式で医療情報が電子化されている.
  主な例を以下に示す.

  \subsubsection{HL7}
  HL7とはHealth Level Sevenの略称である.
  医療情報システム間のISO-OSI第7層アプリケーション層に由来している.
  2015年11月現在,国内で約20の企業が会員となっている.
  特定の部門やシステムに特化したものでなく,施設間・システム間での
  臨床情報や管理情報を扱い,相互運用性を高めるための
  ヘルスケア領域でのデータ交換標準である.\cite{bibi5} \cite{bibi6}

  データ定義は図\ref{ss-mix_sample}のようになっている.
  また,図\ref{ss-mix_sampledata}が
  その出力データのサンプルである.

	\begin{figure}[htbp]
    \begin{center}
			\includegraphics[width=12cm, bb=0 0 792 630]{./gazou/ss-mix_sample.png} %よこたて
    \end{center}
    \caption{HL7のデータ定義}
		\label{ss-mix_sample}
	\end{figure}

	\begin{figure}[htbp]
    \begin{center}
			\includegraphics[width=12cm, bb=0 0 688 437, clip]{./gazou/ss-mix_sampledata.png}
    \end{center}
    \caption{HL7の出力データの例}
		\label{ss-mix_sampledata}
	\end{figure}



  \subsubsection{かかりつけ医が生成するエクセルファイル}
  本研究のために鈴鹿医療科学大学から三重県内の電子カルテを導入していない規模の
  医療機関の診療情報がエクセルのファイルで提供された.



\subsection{本研究の類似製品,活動}
  国内の主な医療情報の共有に関する製品と活動を以下に示す.



  \subsubsection{ID-Link}
    ID-Linkは地域内の病院の患者IDを一元管理することで,
    地域内の病院で作られた電子カルテを参照することができるシステムである.
    医療情報そのものは収集していない.データセンターには患者IDのリンクが
    あるだけで,病院間を安全な通信技術で結び,相互参照させている.
    患者には事前に情報共有に関する許可をもらうことが通例になっている.
    シェアは2015年2月末に全国で4300の機関である.
    \cite{bibi12}

  \subsubsection{SS-MIX}
    SS-MIXは医療情報を収集するために,平成18年から動き出した
    厚生労働省を中心としたプロジェクトである.
    これは標準規格がないまま立ち上がった電子カルテの医療情報の
    電子化についての標準規格である.SS-MIXで規格化された
    基本情報,処方歴,検査結果を各機関のストレージに収集する.
    診断時に医師用端末から参照することや,紹介情報を作るときにも
    情報を引き出すことができる.
    これにはHL7が採用されている.
    \cite{bibi7}

  \subsubsection{あじさいネット}\cite{bibi3}
    2004年に長崎県大村市で始まり、2012年には,
    県域をカバーする地域医療連携ネットワークとして発展してきた.
    2013年4月現在において,電子カルテなどの患者情報の提供を
    行う地域の機関的病院は17病院,地域の診療所や調剤薬局などの
    情報閲覧施設は178施設,医療関係者の会員数は285名を数え,
    これまでに同意を得て登録された患者数は2万6千人を超えている.

    あじさいネットは10年にわたる活動の中でアンケートを繰り返し,
    会費だけで運用することができるシステムになっていった.
    \cite{bibi3}


\section{目的,SQL版,4月に書いたやつまるっと使う}
\subsection{SQL版について}
 医療大から要望があったエクセル形式のデータについてのアプリは
  Djangoで開発した.


\subsection{SQL版の課題とフィードバック}

  開発アプリのデモンストレーションによって得た医療関係者からの意見の中で
  研究課題として任意の検査項目の抽出が挙げられる.

  他の意見はインターフェース寄りの要望が多かった.
  例えば,表によるデータの表示に対するフィードバックとして、

  \begin{itemize}
    \item 任意の検査項目にハイライトをつけてほしい
  \end{itemize}


  今後需要があるであろうバイタルデータの活用に向けて,
  NoSQLを用いたアプリ開発を行う.

  ドキュメントの数だけSQLデータベースのテーブルを用意する必要がある.
  データを検索する際にはjoinしてから.
  比べてNoSQLならガンガン入れて,
  データを出すときにだけKeyの関連づけをすればよい.
  NoSQLならSQLに比べてテーブルを用意する分の
  コストがはぶけてる(と言えるかな).


\section{開発,NoSQL版}
\subsection{アプリケーションの開発環境}
 webアプリケーション開発にはjavascriptのwebフレームワークであるNode.jsを用いた.Node.jsのパッケージであるexpressとnanoを用いた.expressはwebフレームワークで、nanoはCouchDBのためのドライバである.

\begin{table}[htb]
	\begin{tabular}{|l|c|r|r|}\hline
	導入ソフト & ヴァージョン \\ \hline \hline
	Node.js & 0.12.6 \\ \hline
	Express & 4.12.1 \\ \hline
	Passport & 未定 \\ \hline
	\end{tabular}
\end{table}


\subsection{データベースの設計}
	CouchDBにss-mixの仕様書から引っ張ってきたデータ格納方法およびデータ定義\cite{bibi1}に基づいてデータを格納する.CouchDBはひとつのデータベースの中に複数のドキュメントとよばれるデータ構造を保持している.このドキュメントは事前にテーブルなどで定義する必要がない.

	本研究ではひとつの医療行為に対してひとつのドキュメントで管理する.
	ドキュメントが保持する情報を表\ref{tab:doc}に示す.


	\begin{table}[htb]
		\begin{center}
			\caption{ドキュメントが保持する情報}
			\begin{tabular}{|l|c|r|r|}\hline
			Key & Value \\ \hline \hline
			id &  患者名、日付をドキュメントIDとしている. \\ \hline
			rev & \shortstack{ドキュメントの更新回数を示す. \\ 更新時に参照し競合を防ぐ.} \\ \hline
			name & 患者の名前 \\ \hline
			data & \shortstack{医療行為によって得られた情報を \\ json形式で格納.} \\ \hline
			\end{tabular}
			\label{tab:doc}
		\end{center}
	\end{table}

	\begin{figure}[htbp]
		\begin{center}
			\includegraphics[width=5cm, bb=0 0 645 790]{./gazou/ss-mix_sample.png} %よこたて
		\end{center}
		\caption{データ定義}
		\label{ss-mix_sample}
	\end{figure}

	\begin{figure}[htbp]
		\begin{center}
			\includegraphics[width=5cm, bb=0 0 437 688]{./gazou/ss-mix_sampledata.png}
		\end{center}
		\caption{データサンプル}
		\label{ss-mix_sampledata}
	\end{figure}

\subsection{アプリケーションの設計}

	\subsubsection{新出のフォーマットのドキュメントに対するコスト}
	縦向き、横向きのcsv(地域の病院で生まれるような電子化された医療情報)はノーコスト.
	電子カルテ固有の出力ファイルはHL7に対応していればノーコスト.
	json型にもってくまでができれば入力できる.
	出力にはkeyを関連付けるためのコストがかかるが,これは利用者がチューニングしていく.

 %医療情報を収集するNoSQLデータベースシステム.UIとしてWebアプリを用意し,
	%医療関係者,薬剤師,患者の3者に対して,情報を扱いやすいようにした.

\subsection{患者情報閲覧}
	ユーザはログイン後,Accountタブから検索ワードを送信すると,
	/getdbでキーに検索ワードを含むヴァリューを表示する.
	ここで,同義のキーで管理されているヴァリューを表示するために
	キー同士の関連が登録されているドキュメントを参照している.



		\begin{figure}[htbp]
				\includegraphics[width=5cm, bb=0 0 437 688]{./gazou/getdb.png}
			\caption{薬 でデータ抽出した様子}
			\label{ss-mix_sampledata}
		\end{figure}



\subsection{データの投入方法}
	ユーザはログイン後,Input Dataタブを選択する.
	次に入力するファイルを選択し,送信する.\ref{fileiopage}

	\begin{figure}[htbp]
		%\begin{center}
			\includegraphics[width=5cm, bb=0 0 437 688]{./gazou/fileiopage.png}
		%\end{center}
		\caption{ファイル入力ページ}
		\label{fileiopage}
	\end{figure}

	1度の診療で1つのドキュメントを生成する.
	CSV入力ファイルに複数回の診療の記録があることを許容する.


	どうやってCouchからデータを引っ張ってきているか.
	患者のドキュメントを検索してからデータを取得.


		\subsubsection{縦向きcsvファイルの場合}
			parse
			医療大の検査データ
			\\
			\begin{figure}[htbp]
				%\begin{center}
					\includegraphics[width=5cm, bb=0 0 437 688]{./gazou/kensa.png}
				%\end{center}
				\caption{医療大の検査データ}
				\label{ss-mix_sampledata}
			\end{figure}

		\subsubsection{横向きcsvファイルの場合}
			holizontialparse
			医療大の投薬データ
			\\
			\begin{figure}[htbp]
				%\begin{center}
					\includegraphics[width=5cm, bb=0 0 437 688]{./gazou/touyaku.png}
				%\end{center}
				\caption{医療大の投薬データ}
				\label{ss-mix_sampledata}
			\end{figure}

		\subsubsection{パイプ区切りのHL7ファイルの場合}
			parsehl7
			HL7のデータ
			\\
			\begin{figure}[htbp]
				%\begin{center}
					\includegraphics[width=5cm, bb=0 0 437 688]{./gazou/hl7.png}
				%\end{center}
				\caption{HL7のサンプルデータ}
				\label{ss-mix_sampledata}
			\end{figure}



\subsection{同義キーの登録}
	データを参照するときに,キーが必要となる.キーには様々な意味を持つものがあるが,
	異なる規格のデータでは同じ情報を指し示すキーであっても,
	異なるキーが使われている.
	これは新規の規格が医療情報ソフトに流入するたびに課題となる.

	そこで,本研究ではユーザによる同義キーの登録の機能を用意した.
	ユーザは同義である2つのキーを入力すると
	それが同義キーを管理するドキュメントに追加される.

	図\ref{relation}では投薬データの処方日と診断データの日時が同義として登録されている.
	図\ref{relationApp}では検索ワードは処方であるので
	処方をキーに含むデータが検索結果として表示される.
	さらに,検索結果に処方日があり,これは日時と同義として登録されているため,
	日時のデータも検索結果として表示される.
	#実装まだですが.texが嫌になった時にやります。
	#仕様としては、処方日、日時ともにその右に日付がならびます

	\begin{figure}[htbp]
		%\begin{center}
			\includegraphics[width=5cm, bb=0 0 437 688]{./gazou/relation.png}
		%\end{center}
		\caption{同義キーを管理するドキュメント}
		\label{relation}
	\end{figure}

	\begin{figure}[htbp]
		%\begin{center}
			\includegraphics[width=5cm, bb=0 0 437 688]{./gazou/relationApp.png}
		%\end{center}
		\caption{処方と検索して同義キーとして登録されている日時を表示する}
		\label{relationApp}
	\end{figure}


\section{結果}
\subsection{実装した機能とそれによって解決した課題}
  自由に記述されたエクセルファイルと
  入力が想定されている電子カルテの出力ファイルを
  入力ファイルとして受け付けることができた.
  エクセルファイルでは行か列のどちらかに項目があることを
  前提としているので完全に自由とは言えない.
  しかし,エクセルファイルで複数の項目やデータを扱う場合には
  日常的に行か列のどちらかに項目を入力するので
  これは制限にならないと考えられる.
  \\
  様々なフォーマットによって入力された医療情報を
  関連付けて活用するために同義キーの登録機能を実装した.
  これにより同じ意味の項目がフォーマットの都合によって
  消されることなく扱うことができる.
  \\
  患者に認可の権限を与えることで,患者の心的負担を軽減することができる.



\subsection{未解決の課題}
  \subsubsection{ユーザアカウントの管理方法}


  \subsubsection{データの信頼性}
     誰が入力したかをデータと合わせて示したいが,
     海外の先行研究からこれが医療関係者の心理的負担になることがわかっている.
     \cite{bibi10}

 \subsection{企業製品に対する刺激になるといいな}

 \subsection{医療関係者内のお金がらみの事情}
   毎回検査したほうが病院は儲かる.

 \subsection{実装もっと力入れるべきだった}
     SQLver.で実装できてるグラフやガントチャートをNoSQLver.でも使えたらかっこよかった.

 \subsection{実用化にはセキュリティまわりなど課題多し}

 \subsection{本研究の意義}
   ユーザからのデータを入れることができる.
   通院しなくても取れるデータを集めることができる.


\section{考察}
\subsection{企業製品に対する刺激になるといいな}


\appendix

%参考文献

%\begin{thebibliography}{9}
% \bibitem{bibi1} SS-MIX2標準化ストレージ仕様書Ver.1.2c・日本医療情報学会
%  \bibitem{bibi2} 国立病院機構における診療情報分析システムについて・川島直美ら,情報処理学会デジタルプラクティス2013年15号
%  \bibitem{bibi3} 地域医療連携ネットワークの構築と運用継続性の追求・石黒満久
%  \bibitem{bibi4} 「どこでもMy病院」構想の実現 説明資料
%  \bibitem{bibi5} 
%\end{thebibliography}


\begin{thebibliography}{9}
\bibitem{} 中林富雄, 伊藤博文, "地質工学", vol.43, no.6,
  p1131-1136, June, 1997.
\bibitem{} A.Matthes, J.R.Goldsmith, "Natural Juliana", Elsevier
  Publishing, Amsterdam, 1980.
\bibitem{} 桑原裕福, "アメフラシの生態", 旺分社, 横浜, 1999.
\end{thebibliography}
参考文献

\begin{thebibliography}{9}
 \bibitem{bibi1} SS-MIX2標準化ストレージ仕様書Ver.1.2c・日本医療情報学会
  \bibitem{bibi2} 国立病院機構における診療情報分析システムについて・川島直美ら,情報処理学会デジタルプラクティス2013年15号
  \bibitem{bibi3} 地域医療連携ネットワークの構築と運用継続性の追求・石黒満久
  \bibitem{bibi4} 「どこでもMy病院」構想の実現 説明資料
  \bibitem{bibi5}
\end{thebibliography}


\clearpage

\listoffigures
\clearpage

\listoftables
\clearpage


\end{document}
