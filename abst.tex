This paper presents the concept about a web application, in which patients and medical workers can share healthcare information each other.
Healthcare information has not been shared between patients and medical workers in Japan, since the standard for healthcare information systems had not been established.

It is a waste of time and money for patients and medical workers to inspect a patient to get almost same medical information in different hospitals.
Accordingly, it need to develop a sharing web application can receive some different format document.

The proposed system supports CSV format and HL7 format documents as input documents. HL7 is a set of standard for transfer clinical information data between software applications developed by various vender.

The information is stored in CouchDB, which is a type of document database management system using key and value.
Keys, the word of clinical information item, are named as different words by each management systems.
Thus some words have to be used as same semantic keys.
In addition, the same semantic keys must be connected to use as the same semantic data.
This function enables the application to search information that the key have almost same meaning, even if the information made from different softwares.

Finally, the database system was developed by this study.
It can receive some different format data, and collect them.
It may be able to be used in medical surveys.
