\subsection{実装した機能とそれによって解決した課題}
  自由に記述されたエクセルファイルと
  入力が想定されている電子カルテの出力ファイルを
  入力ファイルとして受け付けることができた.
  これにより,統一規格が整備されていない
  医療情報であっても一元的に収集することができると言える.
  ここで,エクセルファイルでは行か列のどちらかに項目があることを
  前提としているので完全に自由とは言えない.
  しかし,エクセルファイルで複数の項目やデータを扱う場合には
  日常的に行か列のどちらかに項目を入力するので
  これは制限にならないと考えられる.

  様々なフォーマットによって入力された医療情報を
  関連付けて活用するために同義キーの登録機能を実装した.
  これにより,同じ意味の項目がフォーマットの都合によって
  消されることなく扱うことができる.

  患者に認可の権限を与えることで,患者の心的負担を軽減することができる.


%\subsection{運用に向けて}
  \if0
  \subsubsection{ユーザアカウントの管理方法}
    患者の電子カルテは病院ごとのIDで管理されているので,
    病院ごとのIDを特定の患者アカウントに対して
    結びつける必要がある.

    また,患者が医療関係者に対して与える認可の範囲において,
    本研究では医師一人に対して権限を与えているが
    その医師が所属する機関に同様の権限を与えるべきかどうか
    検討している.
  \fi

\subsection{データの信頼性}
  %\subsubsection{データの信頼性}
    入力された医療情報はデータを採取した人物や機器の違いを考慮していない.
    これらの差を考慮する必要が出たとき,
    NoSQL版のデータベース内のドキュメントに
    新たな項目を追加することで対応することができる.
    しかし,誰が入力したかを明記して医療情報の共有を行うことは
    医療関係者の心理的負担になることが海外の先行研究から分かっている.
    \cite{bibi10}

%  \subsubsection{データの暗号化}
%    プロトタイプ開発のため,データの保護に関する

\if0
---以下メモ---
企業製品に対する刺激になるといいな
医療関係者内のお金がらみの事情
   毎回検査したほうが病院は儲かる.
実装もっと力入れるべきだった
    SQLver.で実装できてるグラフやガントチャートをNoSQLver.でも使えたらかっこよかった.
実用化にはセキュリティまわりなど課題多し
\fi

\subsection{本研究の意義}
  本研究では国内で規格の統一化が進まない
  医療情報を収集するデータベースシステムと
  それを共有するためのWebアプリケーションの開発を行った.


  ID-Linkでは各病院の患者の電子カルテのIDを関連付けることで,
  あじさいネットでは参照サーバを介して
  電子カルテの情報を参照するだけに留まっている.
  SS-MIXは医療情報を収集するが,患者自身が医療情報を閲覧したり,
  バイタルを追記したりすることはできない.

  提案システムは患者の認可を得ていることを前提に
  患者と医療関係者の間で共有することができるので
  類似製品や活動よりも医療の質の向上を図れると考えられる.

  NoSQL版はSQL版に比べて医療情報の可視化までに
  内部的な処理が増えるが,
  医療情報の形式のばらつきに柔軟に対応させることができた.


  %医療情報の電子化,共有は方針が定まらず,規格を選定している段階にある.
  %今後, 規格が定まった際に
  %多様な規格の元で生成される医療情報を現段階から収集しておけば
  %有効活用できると考えられる.
