\subsection{実装した機能とそれによって解決した課題}
  自由に記述されたエクセルファイルと
  入力が想定されている電子カルテの出力ファイルを
  入力ファイルとして受け付けることができた.
  エクセルファイルでは行か列のどちらかに項目があることを
  前提としているので完全に自由とは言えない.
  しかし,エクセルファイルで複数の項目やデータを扱う場合には
  日常的に行か列のどちらかに項目を入力するので
  これは制限にならないと考えられる.
  \\
  様々なフォーマットによって入力された医療情報を
  関連付けて活用するために同義キーの登録機能を実装した.
  これにより同じ意味の項目がフォーマットの都合によって
  消されることなく扱うことができる.
  \\
  患者に認可の権限を与えることで,患者の心的負担を軽減することができる.



\subsection{未解決の課題}
  \subsubsection{ユーザアカウントの管理方法}


  \subsubsection{データの信頼性}
     誰が入力したかをデータと合わせて示したいが,
     海外の先行研究からこれが医療関係者の心理的負担になることがわかっている.
     \cite{bibi10}

 \subsection{企業製品に対する刺激になるといいな}

 \subsection{医療関係者内のお金がらみの事情}
   毎回検査したほうが病院は儲かる.

 \subsection{実装もっと力入れるべきだった}
     SQLver.で実装できてるグラフやガントチャートをNoSQLver.でも使えたらかっこよかった.

 \subsection{実用化にはセキュリティまわりなど課題多し}

 \subsection{本研究の意義}
   ユーザからのデータを入れることができる.
   通院しなくても取れるデータを集めることができる.
