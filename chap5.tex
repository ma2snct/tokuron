\subsection{考察}
  本研究では国内で規格の統一化が進まない
  医療情報を収集するデータベースシステムと
  それを共有するためのWebアプリケーションの開発を行った.

  ID-Linkでは各病院の患者の電子カルテのIDを関連付けることで,
  あじさいネットでは参照サーバを介して
  電子カルテの情報を参照するだけに留まっている.
  SS-MIXは医療情報を収集するが,患者自身が医療情報を閲覧したり,
  バイタルを追記したりすることはできない.

  提案システムは患者の認可を得ていることを前提に
  患者と医療関係者の間で共有することができるので
  類似製品や活動よりも医療の質の向上を図れると考えられる.

  患者に認可の権限を与えることで,
  患者が自身の医療情報を操作することができるので,
  患者自身の情報が共有されていることによる心的負担を軽減することができると考えられる.
  また,患者からの医療情報の入力を受け付けているので
  提案システムを利用することで健康管理の意識の向上も期待できる.

  自由に記述されたエクセルファイルと
  電子カルテの出力ファイルを
  入力ファイルとして受け付ける
  システムを開発することができた.
  これにより,統一規格が整備されていない
  医療情報であっても一元的に収集することができると言える.

  ここで,エクセルファイルでは行か列のどちらかに項目があることを
  前提としているので完全に自由とは言えない.
  しかし,エクセルファイルで複数の項目やデータを扱う場合には
  日常的に行か列のどちらかに項目を入力するので
  これは制限にならないと考えられる.

  様々な形式によって入力された医療情報を
  関連付けて活用するために同義キーの登録機能を実装した.
  これにより,同じ意味の項目が入力形式の違いによって
  無視されることなく扱うことができる.


\subsection{今後の課題}
  データベースに収集した医療情報をユーザが利用しやすいように
  出力するにはSQLデータベースのほうが扱いやすい.
  本研究ではSQLデータベースを用いたシステムでは
  医療情報をユーザが理解しやすい形で出力できた.
  CouchDBを用いたシステムは複数の形式から
  医療情報を収集するデータベースシステムとなったが,
  ユーザがその医療情報を利用しやすい
  出力を得ることはできていない.
  CouchDBから出力のために必要な医療情報を
  抜き出す処理を追加することで,
  ガントチャートや表を用いて出力することができる.
  具体的には,投薬に関する情報が知りたいときに,
  ガントチャートに出力するために必要な情報を
  CouchDBから取り出して,
  描画処理のプログラムに渡す流れとなる.

  また,入力された医療情報はデータを採取した人物や機器の違いを考慮していない.
  これらの差を考慮する必要が出たとき,
  5章で述べたように,NoSQL版のデータベース内のドキュメントに
  新たな項目を追加することで
  データ入力者や
  機器の情報の入力にも対応することができる.

  しかし,疫学調査のためにデータベースを活用することを想定すると,
  誰が入力したかを明記して医療情報の共有を行うことは
  医療関係者の心理的負担になることが海外の先行研究から分かっているので
  今後のデータベースの利用方法によって調整する必要がある.
  \cite{bibi10}
