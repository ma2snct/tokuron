\subsection{実装した機能とそれによって解決した課題}
   自由に記述されたエクセルファイルと
  入力が想定されている電子カルテの出力ファイルを
  入力ファイルとして受け付けることができた.

  これにより,統一規格が整備されていない
  医療情報であっても一元的に収集することができると言える.
  \\

   ここで,エクセルファイルでは行か列のどちらかに項目があることを
  前提としているので完全に自由とは言えない.
  しかし,エクセルファイルで複数の項目やデータを扱う場合には
  日常的に行か列のどちらかに項目を入力するので
  これは制限にならないと考えられる.
  \\
  様々なフォーマットによって入力された医療情報を
  関連付けて活用するために同義キーの登録機能を実装した.
  これにより,同じ意味の項目がフォーマットの都合によって
  消されることなく扱うことができる.
  \\
  患者に認可の権限を与えることで,患者の心的負担を軽減することができる.



\subsection{運用に際して未解決の課題}
  \subsubsection{ユーザアカウントの管理方法}
    患者の電子カルテは病院ごとのIDで管理されているので,
    病院ごとのIDを特定の患者アカウントに対して
    結びつける必要がある.

    また,患者が医療関係者に対して与える認可の範囲において,
    本研究では医師一人に対して権限を与えているが
    その医師が所属する機関に同様の権限を与えるべきかどうか
    検討している.


  \subsubsection{データの信頼性}
     誰が入力したかをデータと合わせて示したいが,
     海外の先行研究からこれが医療関係者の心理的負担になることがわかっている.
     \cite{bibi10}

---以下メモ---
企業製品に対する刺激になるといいな
医療関係者内のお金がらみの事情
   毎回検査したほうが病院は儲かる.
実装もっと力入れるべきだった
    SQLver.で実装できてるグラフやガントチャートをNoSQLver.でも使えたらかっこよかった.
実用化にはセキュリティまわりなど課題多し

 \subsection{本研究の意義}
   ユーザからのデータを入れることができる.
   通院しなくても取れるデータを集めることができる.
