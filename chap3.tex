\subsection{SQL版について}
 医療大から要望があったエクセル形式のデータについてのアプリは
  Djangoで開発した.


\subsection{SQL版の課題とフィードバック}

  開発アプリのデモンストレーションによって得た医療関係者からの意見の中で
  研究課題として任意の検査項目の抽出が挙げられる.

  他の意見はインターフェース寄りの要望が多かった.
  例えば,表によるデータの表示に対するフィードバックとして、

  \begin{itemize}
    \item 任意の検査項目にハイライトをつけてほしい
  \end{itemize}


  今後需要があるであろうバイタルデータの活用に向けて,
  NoSQLを用いたアプリ開発を行う.

  ドキュメントの数だけSQLデータベースのテーブルを用意する必要がある.
  データを検索する際にはjoinしてから.
  比べてNoSQLならガンガン入れて,
  データを出すときにだけKeyの関連づけをすればよい.
  NoSQLならSQLに比べてテーブルを用意する分の
  コストがはぶけてる(と言えるかな).
